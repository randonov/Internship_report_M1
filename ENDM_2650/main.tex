\documentclass{endmH}
\usepackage{endmmacro}
\usepackage{graphicx}
\usepackage{subfigure}
\usepackage{color}

% The following is enclosed to allow easy detection of differences in
% ascii coding.
% Upper-case    A B C D E F G H I J K L M N O P Q R S T U V W X Y Z
% Lower-case    a b c d e f g h i j k l m n o p q r s t u v w x y z
% Digits        0 1 2 3 4 5 6 7 8 9
% Exclamation   !           Double quote "          Hash (number) #
% Dollar        $           Percent      %          Ampersand     &
% Acute accent  '           Left paren   (          Right paren   )
% Asterisk      *           Plus         +          Comma         ,
% Minus         -           Point        .          Solidus       /
% Colon         :           Semicolon    ;          Less than     <
% Equals        =           Greater than >          Question mark ?
% At            @           Left bracket [          Backslash     \
% Right bracket ]           Circumflex   ^          Underscore    _
% Grave accent  `           Left brace   {          Vertical bar  |
% Right brace   }           Tilde        ~

\newcommand{\HRule}{\rule{\linewidth}{0.5mm}}
\newcommand{\forLaterVersions}[1]{}

\def\CC {{\mathbb C}}        
\def\RR {{\mathbb R}}       
\def\ZZ {{\mathbb Z}}      
\def\NN {{\mathbb N}}     

\newcommand{\Nat}{{\mathbb N}}
\newcommand{\Real}{{\mathbb R}}
\def\lastname{Please list your Lastname here}

\begin{document}

% DO NOT REMOVE: Creates space for Elsevier logo, ScienceDirect logo
% and ENDM logo
\begin{verbatim}\end{verbatim}\vspace{2.5cm}

\begin{frontmatter}

\title{Global Optimization for Scaffolding and Completing Genome Assemblies }\thanks{Preliminary version of this paper has been presented at the Workshop on Constraint-Based Methods for Bioinformatics 2016.}

\author{Sebastien Fran\c{c}ois  \thanksref{sebemail}}
\author{Rumen Andonov \thanksref{rumemail}\thanksref{Corrresponding}}
\author{Dominique Lavenier \thanksref{domemail}}
\address{IRISA/INRIA, Rennes, France}

\author{Hristo Djidjev \thanksref{hriemail}}
\address{Los Alamos National Laboratory,
	Los Alamos, NM 87545, USA }
   \thanks[sebemail]{Email:
   \href{mailto:sefra35@gmail.com} {\texttt{\normalshape
   sefra35@gmail.com}}} 
   \thanks[rumemail]{Email:
   \href{randonov@irisa.fr} {\texttt{\normalshape
   randonov@irisa.fr}}}
   \thanks[Corrresponding]{{\texttt{\normalshape Corresponding author }}} 
   \thanks[domemail]{Email:
   \href{lavenier@irisa.fr} {\texttt{\normalshape
   lavenier@irisa.fr}}}
   \thanks[hriemail]{Email:
   \href{djidjev@lanl.gov} {\texttt{\normalshape
   djidjev@lanl.gov}}}
   

\begin{abstract}
We develop a method for solving genome scaffolding  as a problem of finding a long simple path in a graph defined by the contigs that satisfies additional constraints encoding the insert-size information. Then we solve the resulting mixed integer linear program to optimality using the Gurobi solver. We test our algorithm on several chloroplast genomes and show that it %is fast and 
outperforms other widely-used assembly solvers by the accuracy of the results.
\end{abstract}

\forLaterVersions
{\begin{keyword}
genome assembly, scaffolding, contig, longest  simple weighted  path problem,
integer programming  
\end{keyword}}

\end{frontmatter}

\section{Introduction}

Modern Next-Generation Sequencing  (NGS) techniques  are not able to output the whole genome sequence in one large sequence, but instead output billions of short DNA sequences, called \textit{reads}.  These reads  are extremely redundant, erroneous, and, consequently, unusable  practically. \textit{Genome assembly} is the challenging computational task, consisting in reconstructing the full genome sequence from these fragmented raw data. This  is a complex procedure, usually composed of three main steps: (1) generation of \textit{contigs}, which are long contiguous DNA sequences issued from the overlapping of the reads; (2) orientation and ordering of these contigs, called the \textit{scaffolding}; (3) gap-filling. A notable weakness of the contemporary approaches for \emph{de-novo assembly} is to consider the above process as a set of independent tasks and not be able to propose a global optimal solution.

The first step generates a list of \textit{contigs} (assembled ungapped sequences) that represent the "easily assembled regions"  of the genome.  Building contigs is currently supported by methods using a specific data structure called  \emph{de-Bruijn} graph \cite{Pevzner14082001}.
\forLaterVersions{, whose vertices  are all $k$-mers ($k$-length subwords of the reads) and whose edges connect all pairs of $k$-mers that share $k-1$ consecutive characters. Genomes are then sought as maximal unambiguous paths in the de Bruijn graph.
A lot of work has been done on this topic and the community profits today from  the development of efficient contig generators % and a very compact structure for representing de-Bruijn graphs with Bloom filters 
\cite{minia,others??}.  
However, complex regions of the genome (i.e. regions with many repeats) generally fail to be assembled by this technique: if there are repeats (identical subregions of the genome) longer than the size of the reads, the entire genome cannot be built in a unique way. }%Various heuristics are used to bypass simple repeats, but they do not guarantee correct solution
%

Whereas the main challenge in the first step is in the sheer size of the raw data,  in the second step, scaffolding, the data is of moderate size, but the problem remains largely open  because of its NP-hard complexity \cite{huson_greedy_2002}).  The goal here is to provide a reliable order and orientation of the contigs in order to link them together into \textit{scaffolds}-- a sequence of contigs that overlap or are separated by gaps of a given length. The gap information is generated during sequencing based on \textit{paired-end} or \textit{mate-pair} reads \cite{Weber01051997\forLaterVersions{,Medvedev_11}}, and has distance information associated with the gap. Specifically, each gap can be represented as a couple of fragments separated by a known distance (called \textit{insert size}) and provides information about the distance between the corresponding contigs. 
%They bring a long distance information that is missing in the first step and that can be %used for connecting contigs generated by de-Bruijn graphs.   

%Most previous work on scaffolding is heuristics based, e.g., SSPACE~\cite{boetzer_scaffolding_2011}, GRASS~\cite{Gritsenko01062012}, and BESST~\cite{BESST}. These scaffolders may find %in some cases  good solutions, but their accuracies cannot be guaranteed or predicted.  As far as we know, no method exists today that truly models the entire contigs relationship  and exactly solves  the underlying optimization problem.% in case of large and complex genomes \cite{Bosi25032015,ilp_montpellier}. 

\forLaterVersions{In addition to be NP-hard,  the scaffolding is challenging  because of technology  deficiencies like:
\begin{itemize}
	\item 
	Insert sizes are not precise. The technology provides approximate distance information only.
%	\item Distances between contigs may be larger than the insert size. In that case, scaffolding the whole genome is not possible. The solution will be a set of scaffolds;
%	\todo{this paragraph has been commented out. I am not sure how we recognize this case--we know the insert sizes, but not the distances. Then how do we decide whether to connect the scaffolds or keep them separated? In any case, we do not produce a set of scaffolds.}
	\item For short contigs, usually related to short repeat regions, multiplicity (the repeat factor) cannot be precisely determined. %This information is given by analyzing the coverage. 
	Shorter the contig, worse the estimation.
	\item There may be erroneous contigs. Heuristics implemented for generating contigs may lead to chimeric sequences that wrongly connect two regions of the genome.
\end{itemize}
}

While the ultimate goal of the genome assembly is to generate a complete genome, the scaffolding phase usually produces a set of multiple scaffolds that, in addition, may contain inside them regions that have not been completely predicted. %For example, for two contigs that have been unambiguously  linked,  the nucleotides sequence between them may have not been determined due to sequencing problem, or very high structure complexity. 
Further stages, such as  \textit{gap-filling} and  a step that we call here  \textit{scaffold extension} (elongating and concatenating the contigs after  the scaffolding step) are needed  to complete the genome. %enhance the scaffold. 

This paper focuses simultaneously on the above mentioned three  steps (scaffolding, gap filling and scaffold extension)  of de nouveau genome assembly.  Given a set of contigs and their relationships--overlaps and/or remoteness  in terms of distances between them (insert sizes)--we propose a global  optimization-based approach for completing  the genome assembly as the longest sequence that is consistent with the given contigs and linkage information. 
A drawback of the typically used strategy of constructing a set of disjoint paths, rather than a single path, is that it would require additional steps of gap filling and scaffold extension, involving additional work. Moreover, it would make impossible to find a provably optimal  final solution, since, even if each separate problem is implemented optimally, their combination may not be optimal.


Here we introduce a so called   \textit{contig graph}, that encodes information about contigs and distances between them,
%whose vertices are the contigs and whose edges connect pairs of contigs that either overlap, or have a gap of size given by the insert-size information. Edges have weights that encode the corresponding distance information between the contigs and are negative in the case of overlaps and positive in the case of gaps. Vertices have weights equal to the lengths of the contigs they represent. Contigs with repeat factor $s$ are represented as a set of $s$ vertices with the same sets of neighbors.  The length of a path in the resulting graph is defined as the sum of the weights of the vertices and edges in it.  
and reduce the  scaffolding and gap-filling  to finding a longest simple path in that graph such that as many as possible mate-pairs distances are satisfied  (we call hereafter such path just a \textit{longest path}).  Since both conditions cannot generally be simultaneously satisfied, our objective function is a linear combination of them.  We solve this problem by reformulating it as a mixed integer linear program (MILP) and develop a method that  exactly solves the resulting program on genomes of up to 165 contigs and up to 6682  binary variables. We analyze the performance of the algorithm on several chloroplast genomes and compare it to other scaffolding algorithms. 

An advantage of our approach is that the modeling of scaffolding as a longest path problem allows one to solve simultaneously all % several 
subtasks specific for completing the genome assembly. %  like: contig orientation and ordering, repeats,  gap filling, and scaffold extension, which in other approaches are targeted as separate problems. 
We are not aware of previous approaches on scaffolding based on the longest path problem reduction.  There is no guarantee that the genome sequence corresponds to a longest path, but our experiments show that that is the case in many instances or, if not, there is a very small difference between the two.  Unlike the shortest path problem with non-negative weights, for which efficient polynomial-time algorithms exist,
the longest weighted path problem is NP-hard \cite{Garey:1990:CIG:574848}, which means that no polynomial time solution is likely to exist for it. 

We tested this model on a set of chloroplast and bacteria  genome data  and showed that it allows  to assemble the complete genome as a single scaffold. Compared to the publicly available scaffolding tools that we have tested, our solution produces assemblies of significantly higher quality. 

Most previous work on scaffolding is heuristics based, e.g., SSPACE~\cite{boetzer_scaffolding_2011}, GRASS~\cite{Gritsenko01062012}, and BESST~\cite{BESST}. Such algorithms may find in some cases good solutions, but their accuracies cannot be guaranteed or predicted.  %In contrast, our method always finds a longest path in the contig graph. 
Exact algorithms for the scaffolding problem are presented in~\cite{weller2015exact}, but the focus of that work is on finding structural properties of the contig graph that will make the optimization problem of polynomial complexity.  In \cite{ilp_montpellier}, integer linear programming is used to model the scaffolding problem, with an objective  to maximize the number of links that are satisfied.  In order to avoid sub-cycles in the  solution, the authors use 
%do not use subcycle elimination constraint, but solve the problem 
 an incremental process, where cycles that may have been produced by the solver are forbidden in the next iteration.   %It is difficult to compare their approach with ours.  
 While our focus is on accuracy,  \cite{ilp_montpellier} focuses on efficiency, and indeed their algorithm, being a kind of heuristics, is faster than ours. 
However,  integrating the distances between contigs  and  accounting for possible multiplicities of the contigs (repeats) is indicated  as future improvement in 
 \cite{ilp_montpellier},  while it has been realized in our approach. 
 
%In contrast, our objective is to maximize the length of the resulting scaffold. Moreover, we aim at producing a single path or cycle, rather than a set of paths and cycles. %We believe that by requiring our solution to be a single path we avoid the risk of producing a set of paths for which the objective function is of  high value, but which are inconsistent with a single path ordering. 


\smallskip
The contributions  of this study are as follows:\vspace*{-0.2cm}
\begin{itemize}
\item   Our modeling of the scaffolding problem as a longest path problem allows to solve \emph{simultaneously} the set of subtasks specific for %this problem 
completing the genome assembly  like: contigs orientation and ordering, repeats,  gap filling and scaffold extension, which in other approaches are separate phases. 
\item  The  scaffolding problem  is reduced  to finding a longest path in a particular graph. In addition, these paths need to satisfy a set of distances between  couples of vertices along these paths. We are not aware of previous approaches on scaffolding based on the longest path problem. 
%The disadvantage of having a set of disjoint paths, rather than a single path, is that it would require additional steps of gap filling and scaffold extension, involving additional work. Moreover, it would make it impossible to produce an optimal  final solution, since, even if each separate problem is implemented optimally, their combination may not be optimal. 

\item   We formulate the above problem as a mixed integer linear program (MILP) with several interesting properties like:  cycles elimination constraints and using binary variables for the edges of the graph only.  Vertices are modeled with real variables, but we prove that the integrality of these variables follows from other constraints. Moreover,  the commonly used approach for solving the longest weighted simple path when the initial (source) and final (target) vertices are unknown consists in artificially adding these two vertices and $2|V|$ edges in the graph $G=(V,E)$ \cite{Bui2016}. This increases by $2|V|$ the number of binary variables, which is a drawback when the density of the graph is small (as is the case of scaffolding graph). In contrast, our modeling does not require such a graph transformation and requires fewer binary variables.

\item   We tested this model on a set of chloroplast and bacteria  genome data  and showed that it allows  to assemble the complete genome as a single scaffold. None of the publicly available scaffolding tools that we have tested targets  single scaffolds (this is corroborated by the obtained numerical results). 
\item Our numerical experiments indicate that the relaxation of the mixed integer model is tight and produces upper bounds of excellent quality. This suggests a promising direction of research towards the scalability of our approach.

\end{itemize}

\forLaterVersions{In the next Section 2 we describe our graph model and the formulation of the optimization problem and in Section 3 we present experimental results and comparison with other algorithms.}

\label{sec:generalities}



\section{Modeling the scaffolding problem}\label{Model description}
\subsection{Graph Modeling}\label{sub:Graph_Modelling}



We model the problem of scaffolding as path finding in a directed graph $G=(V,E)$ that we call a contig graph, where  both  vertices $V$ and  edges $E$ are weighted. The set of vertices $V$ is generated  based on the set $C$ of the contigs according the following rules:  the contig $i$ is represented by at least  two vertices $v_i$ and $v'_i$ (forward/inverse  orientation respectively).  If the contig $i$ is repeated $k_i$ times, it generates $2k_i$ vertices.  %The values of $k_i$ are computed in the contigs-generation phase.
Denote $N= \sum_{i\in C} k_i$, therefore $|V|=2N$.

 The edges are generated following given patterns---a set of known overlaps/distances between the contigs. 
 Any edge is given in the graph $G$ in its forward/inverse orientation. We denote by $e_{ij}$ the edge joining vertices $v_i$ and $v_j$ and the inverse of
  edge  $e_{ij}$ is  $e_{j^{'}i^{'}}$.
 For any $i$, the weight $w_i$ on a vertex $v_i$ corresponds to the length of the contig $i$, while the weight $l_{ij}$ on the edge  $e_{ij}$  corresponds to the value of the  overlap/distance between  contigs $i$ and $j$.   The problem then is to find a path in the graph $G$ such that the total  length (the sum over the traversed vertices and edges) is maximized, %as close as possible to a given distance $L$,  
 while a set of additional constraints are also satisfied:
\begin{itemize}
%  \item The first and the last vertices (denoted by $s$ and $t$ respectively) are known. They are  visited exactly once, while all other nodes are visited at most once. 
  \item For any $i$, either vertex $v_i$ or $v'_i$ is visited  (participates in the path).
  \item The orientations of the nodes does not contradict the constraints
    imposed by  mate-pairs. This is at least partially enforced by the
    construction of $G$. 
\end{itemize}



To any edge $e\in E$ we associate a variable $x_e$.  Its value is set to 1, if the corresponding  edge participates in the assembled genome sequence (the associated path in our case), otherwise its value is set to 0. 
 There are two kinds of edges: edges corresponding to overlaps between contigs, denote them by $O$ (from overlaps), and edges associated with mate-pairs relationships, denote them by $L$ (from  links).  We therefore  have $E=L \cup O$.  Let $l_e$ be the length of the edge $e=(u,v)$.  We have $l_e < 0$ and $|l_e|< \min{\{w(u),w(v)}\}$,  $\forall e \in O$, and  $l_e > 0  ~\forall e \in L$.  Let  $w_v$ be the length of the contig corresponding to vertex $v$ and denote $W= \sum_{v \in V} w_v$. %A vertex $v$ is called \emph{big} if $w_v > l_e ~~\forall e \in L$ holds. Let $B$ denote the set of big vertices.  The set of others, \emph{small}, vertices  is denoted by $S$. We therefore have  $V=B\cup S$. 

%The set of vertices  can be partitioned in three subsets $V=S \cup T \cup U$ such that each $v\in S$ is a head of some $e\in L$, and each $v\in T$ is a tail of some $e\in L$.

%\newpage 

Let $A^+(v) \subset E $ (resp. $A^-(v) \subset E  $ ) denote  the subset  of arcs in $E$ leaving (resp. entering)  node $v$. 

\subsection{Mixed Integer Linear Programming Formulation} % (fold)
\label{sub:first step model}

                   
We  associate a binary variable for any edge of the graph, i.e.
   \begin{equation}\label{binary_edges}
    \forall e \in O:  x_e \in \{0,1\} \mbox{ and } \forall e \in L:  g_e \in \{0,1\}. 
   \end{equation}
   
 Furthermore, to any vertex $v\in V$ we associate three variables,  $i_v$, $s_v$, and $t_v$, which stand respectively for intermediate, source, and target for some path, and satisfy

  \begin{equation}\label{reals}
 %\forall  v\in V :  
  0 \leq i_v \leq 1, ~~ 0 \leq s_v \leq 1, ~ 0 \leq t_v \leq  1.
 \end{equation}
 
 All three variables are set to zero when the associated vertex $v$ participates in none of the paths. 
 Otherwise, $v$ can be  either a source/initial (noted by $s_v=1, t_v=0, i_v=0$), or a target/final ($t_v=1,  s_v=0, i_v=0$), or an intermediate vertex, in which case the equalities $i_v=1, t_v=0$ and $s_v=0$ hold.  Moreover, each vertex (or its inverse) can  be visited  at most once, i.e.
                   \begin{equation}\label{at_most_one}
                   \forall (v,v') :  i_v+i_{v'} + s_v+s_{v'} + t_v+t_{v'} \leq 1.
                   \end{equation}


The four possibles states for a vertex $v$ (to belong to none of the paths, or otherwise, to be a source, a target,  or an intermediate  vertex in some path) are provided by the following  two constraints 

 \begin{equation}\label{output-edges}
s_v + i_v =  \sum_{e\in A^{+}(v)} x_e,  %\leq 1,  %\mbox{ and } \sum_{e\in A^{-}(v)} x_e  \leq  i_v+t_v 
% \end{equation}
% 
% and 
\quad
% 
%  \begin{equation}\label{input-edges}
t_v  + i_v = \sum_{e\in A^{-}(v)} x_e. %\leq 1.  %\mbox{ and } \sum_{e\in A^{+}(v)} x_e  \leq  i_v + s_v .
  \end{equation}
  
    
   Finally, only one sequence (a single path) is searched for 
    
        \begin{equation}\label{one_paths}
      \sum_{v \in V} s_v = 1  \mbox{ and }  \sum_{v \in V} t_v = 1.
       \end{equation}
                  
           
  
    \begin{theorem}\label{vertices_are_binary}
    The real variables $ i_v, s_v, t_v, \forall v \in V$ take binary values.
    \end{theorem}
   \begin{proof} Given in \cite{RR_9050}.
   \end{proof}
                      
   We introduce a continuous variable $f_e \in  R^+$ to express the quantity of the flow circulating along the arc $e \in E$
%                             
  \begin{equation}\label{flow_condition}
  \forall e \in E:  0 \leq f_e \leq Wx_e.   %\in \{0,1\} 
  \end{equation}
                                
                     
  
  For $e\in O$, the value of $x_e$ is set to  $1$, if the arc $e$ carries some  flow and $0$, otherwise. In other words, no  flow can use  the arc $e$ when $x_e = 0$. \forLaterVersions {as ensured by constraint} % (\ref{flow_capacity}).
   %  
\forLaterVersions{
       \begin{equation}\label{flow_capacity}
    f_e \leq W x_e  ~~~~ \forall e \in O.
       \end{equation}
}
       
 We use the flows $f_e$ in the following constraints, $\forall v\in V$,
         \begin{equation}\label{no_cycles}
  \sum_{e\in A^{-}(v)} \!\!f_e  -  \!\!\sum_{e\in A^{+}(v)} \!\!f_e   \geq   (i_v + t_v) (w_v +   \!\!\sum_{e\in A^{-}(v)} \!\!l_e x_e) - W s_v, %+ t_v\sum_{e\in A^{-}(v)} f_e       
\quad
    W s_v \leq    \!\!\sum_{e\in A^{+}(v)} \!\!f_e. %\leq W
    \end{equation}

    
The purpose of the last  two constraints is manifold.  When a vertex $v$ is a source ($s_v=1$), (\ref{no_cycles}) \forLaterVersions{and (\ref{initial_flow})} generates and outputs from it an initial flow of sufficiently big value ($W$ is enough in our case). When $v$ is an intermediate vertex  ($i_v=1$), constraint (\ref{no_cycles}) forces the flow to decrease by at least $l_{(u,v)}+w_v$  units when it moves from vertex $u$ to its adjacent vertex $v$. The value of the flow thus is decreasing and this feature forbids cycles in the context of (\ref{output-edges}).  When $v$ is  a final vertex, (\ref{no_cycles}) is simply a valid inequality for the input flow.% since the initial flow value is big enough. 
  
    
We furthermore observe that because of (\ref{output-edges}), the constraint   (\ref{no_cycles}) can be written as follows 
             \begin{equation}\label{no_cycles_linear}
      \forall v\in V:\sum_{e\in A^{-}(v)} f_e  -  \sum_{e\in A^{+}(v)} f_e   \geq    (i_v+t_v) w_v +   \sum_{e\in A^{-}(v)} l_e x_e - W s_v. %+ t_v\sum_{e\in A^{-}(v)} f_e
             \end{equation}
             
The constraint  (\ref{no_cycles_linear}) is  linear and we keep it in our model instead of  (\ref{no_cycles}).       


Furthermore, binary variables $g_e$ are associated with links. For $(s,t) \in L$, the value of $g_{(s,t)}$ is set to  $1$ only if  both vertices $s$ and $t$ belong to the selected path and the length of the considered path between them is in the given interval  $[\underline{L}_{(s,t)},\overline{L}_{(s,t)}]$. Constraints related to links are :
%
 \begin{equation}\label{gst_left}
 g_{(s,t)} \leq s_s+i_s+t_s \mbox{ and }  g_{(s,t)} \leq s_t+i_t+t_t
       \end{equation}
%       as well as
  \begin{equation}\label{lbound}
  \forall (s,t) \in L: 
  \sum_{e\in A^{+}(s)} f_e - \sum_{e\in A^{-}(t)} f_e  \geq   \underline{L}_{(s,t)} g_{(s,t)} - M (1-g_{(s,t)})
       \end{equation}
         \begin{equation}\label{upbound}
  \forall (s,t) \in L: 
 \sum_{e\in A^{+}(s)} f_e - \sum_{e\in A^{-}(t)} f_e  \leq   \overline{L}_{(s,t)} g_{(s,t)}) + M (1-g_{(s,t)}),
       \end{equation}
%       
where $M$ is some big constant.  
         
We search for a long path in the graph and such that as much as possible  mate-paired distances  are satisfied. The objective hence is : 
%
 \begin{equation}\label{all_coeff}
 \max \left( \sum_{e\in O} x_e l_e  +   \sum_{v\in V} w_v ( i_v+ s_v + t_v ) + p \sum_{e\in L} g_e   \right)
\end{equation}
where $p$ is a parameter to be chosen as appropriate (currently $p=1$). 

\forLaterVersions{\smallskip\textbf{Remark:} Note that omitting constraints  (\ref{one_paths}) %and keeping the objective and all other constraints in 
from the above model generates a set of paths that cover  "optimally" the contig graph, rather than a single path.  We have tested this variant of the model, but the obtained solutions were too much fragmented and of worse  quality compared to the single-path model.}

\section{Computational results}

Here we present the results obtained on a small set of chloroplast and bacteria  genomes given in Table \ref{tab:data}.  Synthetic sequencing reads have been generated  for these instances applying ART simulator \cite{Huang15022012}. For the read assembly step required to produce contigs  we applied  the well-known \textnormal Minia \cite{minia} with  parameter  {\tt unitig} instead of {\tt contig} (a unitig is a special  kind of a high-confidence contig).
Minia generates  the set of  unitigs, their repetition factor (the value of $k_i$),  the overlaps between them, as well as the  mate-pair edges and distances.  Based on this data, we generate a graph as explained in Section \ref{sub:Graph_Modelling}. 

%  these unitigs, the overlaps between them, as well as the mate-pair distances, we generated a graph as explained in Section \ref{sub:Graph_Modelling}. 
\forLaterVersions{The graph generated for  the  Atropa belladonna  genome  is given in  Figure~\ref{fig:Atropa_graph}.}

% The chloroplast genomes are relatively small, but have a particularity that makes them interesting as data for scaffolding. As shown in \cite{Kolodner:1979kx},  chloroplast genomes present an inverted repeat region of approximately 20kbp. We visualize such a repeat region  for  the solution  obtained  for the Atropa belladonna  genome in Figure~\ref{fig:Acorus_solution}.  
 
\forLaterVersions{
  \begin{figure}[htp]
      \centering
      \includegraphics[height=0.6\textwidth,width=0.9\textwidth]{figs/probleme_atropa}
      \caption{The contig graph  generated for the Atropa belladonna genome. Red/blue vertices correspond respectively to big/small contigs. }
      \label{fig:Atropa_graph}
  \end{figure}
}
  
\forLaterVersions{
    \begin{figure}[p]
      \centering
      \includegraphics[height=0.5\textwidth,width=0.9\textwidth]{figs/sous_graphe_atropa.eps}
      \caption{The reduced contig graph  generated for the Atropa belladonna genome. It contains only red/big vertices and dashed/mate-paires edges. }
      \label{fig:reduced_Atropa_graph}
  \end{figure}
}   
   
\forLaterVersions{
   \begin{figure}[htp]
      \centering
 \begin{minipage}[c]{.45\linewidth}
 \begin{center}
 \includegraphics[width=5.5cm,height=6cm]{figs/solution_step1_atropa.eps} 
\end{center}
 \end{minipage}
 \begin{minipage}[c]{.5\linewidth}
\includegraphics[width=5.5cm,height=6cm]{figs/solution_atropa.eps}  
\end{minipage}
}

\forLaterVersions{
      \centering
      \includegraphics[height=0.4\textwidth,width=0.9\textwidth]{figs/solution_atropa}
     \caption{The scaffold obtained for Atropa belladonna's genome\forLaterVersions{: ~left--the solution for the reduced contig graph from  Figure~\ref{fig:reduced_Atropa_graph}; ~right-- the previous solution has been extended to the entire  Atropa belladonna genome by  solving the original problem from} shown on Figure~\ref{fig:Atropa_graph}. }
     \label{fig:Atropa_solution}
\end{figure}  
}

% \begin{figure}[p]
%     \centering
%     \includegraphics[height=0.5\textwidth,width=0.8\textwidth]{AcorusSolution.png}
%     \caption{The scaffold obtained from the graph on  Figure~\ref{fig:Acorus_graph}. Yellow vertices indicate the zone of inverted repeats. }
%     \label{fig:Acorus_solution}
% \end{figure}
 

Our results were obtained on an  Intel(R) Xeon(R) CPU E5-2670 v2 @ 2.50GHz with 20 cores, 64 GB of RAM, and using Gurobi 6.5.1 solver for solving the MILP models. We compared our results with the results obtained by three of the most recent scaffolding tools--  SSPACE \cite{boetzer_scaffolding_2011},  BESST \cite{BESST}, and Scaffmatch  \cite{Mandric17042015}.
In order to evaluate the   quality of  the produced scaffolds, we applied the QUAST tool  \cite{gurevich_quast:_2013}.
The results are shown on Table~\ref{tab:tools}.
 We observe that our tool GST (from Genscale Scaffolding Tool) is the only one that consistently  assembles the complete genome (an unique scaffold in \#scaffolds column) with more than 98\% (and in four cases at least 99.9\%) correctly predicted genome fraction and zero misassembles. 
 
 

       
 
      \begin{table}
\footnotesize
          \centering
              \begin{tabular}{|c|c|r|r|r|r|} 
              \hline
          Datasets & Total & \#unitigs  & \#nodes & \#edges   & \#mate-pairs \\ 
%            & length &   &  &   & pairs  \\
               \hline
              \hline
              Acinetobacter
              & 3 598 621 & 165 & 676 & 8344 & 4430 \\ 
              \hline
              \hline
              Wolbachia 
              & 1 080 084 & 100 & 452 & 7552 & 2972 \\ 
              \hline
              \hline
              Aethionema 
%              &  &  &  &  &  \\ 
              Cordifolium
              & 154 167 & 83 & 166 & 898 & 600 \\ 
              \hline     
              \hline
              Atropa belladonna
              & 156 687 & 18 & 36 & 114 & 46 \\ 
              \hline
              \hline
              Angiopteris Evecta
              & 153 901  & 16 & 32 & 144 & 74 \\ 
              \hline
              \hline
              Acorus Calamus
              & 153 821  & 15 & 30 & 134 & 26 \\ 
               \hline                       
              \end{tabular}
             \caption{Scaffolding datasets. }
              \label{tab:data}
            \end{table}
            
             
      
 \begin{table}
%\tiny \footnotesize 
\small
    \centering
        \begin{tabular}{|c|c|r|r|r|r|} 
        \hline
    Datasets & Scaffolder & Genome  & \#sca- & \# misass-   & N's per  \\
       &  & fraction & ffolds & emblies  &  100 kbp \\
         \hline
         \hline
         Acinetobacter
         & GST & 98.536\% & 1 & 0 & 0 \\ 
         \hline   
         & SSPACE & 98.563\% & 20 & 0 & 155.01 \\ 
         \hline
         & BESST & 98.539\% & 37 & 0 & 266.65 \\ 
         \hline
         & Scaffmatch & 98.675\% & 9 & 5 & 1579.12 \\ 
         \hline
          \hline
          Wolbachia
          & GST & 98.943\% & 1 & 0 & 0 \\ 
          \hline   
          & SSPACE & 97.700\% & 9 & 0 & 2036.75 \\ 
          \hline
          & BESST & 97.699\% & 49 & 0 & 642.90 \\ 
          \hline
          & Scaffmatch & 97.994\% & 2 & 2 & 3162.81 \\ 
          \hline
          \hline
           Aethionema 
          &  & &  &  &  \\ 
          Cordifolium
          & GST & 100\% & 1 & 0 & 0 \\ 
          \hline   
          & SSPACE & 95.550\% & 20 & 0 & 13603.00 \\ 
          \hline
          & BESST & 81.318\% & 30 & 0 & 1553.22 \\ 
          \hline
          & Scaffmatch & 82.608\% & 7 & 7 & 36892 \\ 
          \hline
         \hline
         Atropa belladonna
         & GST & 99.987\% & 1 & 0 & 0 \\ 
         \hline   
     & SSPACE & 83.389\% & 2 & 0 & 155.01 \\ 
         \hline
         & BESST & 83.353\% & 1 & 0 & 14.52 \\ 
         \hline
         & Scaffmatch & 83.516\% & 1 & 0 & 318.93 \\ 
         \hline
         \hline
         Angiopteris Evecta
         & GST & 99.968\% & 1 & 0 & 0 \\ 
         \hline   
         & SSPACE & 85.100\% & 4 & 0 & 0 \\ 
         \hline
         & BESST & 85.164\% & 2 & 0 & 1438.54 \\ 
         \hline
         & Scaffmatch & 85.684\% & 1 & 0 & 454.23 \\ 
         \hline
                  \hline
                  Acorus Calamus
         & GST & 100\% & 1 & 0 & 0 \\ 
         \hline   
         & SSPACE & 83.091\% & 4 & 0 & 126.39 \\ 
         \hline
         & BESST & 83.091\% & 4 & 0 & 127.95 \\ 
         \hline
         & Scaffmatch & 83.271\% & 1 & 1 & 3757.13 \\ 
         \hline              
        \end{tabular}
       \caption{Performance of different solvers on the datasets from Table~\ref{tab:data}. GST is our tool. }
        \label{tab:tools}
      \end{table}
      
      
     
      
 Our next computational experiments focussed on comparing various relaxations and other related formulations for the above described models. 
 Due to lack of space these results are not presented here, but the interested reader can find them  in \cite{RR_9050}.
 \forLaterVersions{ 
 (the computational results  are presented in \cite{RR_9050}) . 
 %described in the previous section. 
 Denote %in the sequel  
 by  BR (Basic Real) the model  defined by  %the linear
  constraints (\ref{binary_edges}-\ref{flow_condition}) and (\ref{no_cycles_linear}-\ref{upbound})
%    (\ref{binary_edges}), (\ref{reals}), (\ref{at_most_one}),  (\ref{output-edges}),  (\ref{input-edges}), (\ref{one_paths}),  (\ref{flow_condition}), %(\ref{flow_capacity}), 
%    (\ref{no_cycles_linear}), 
%    %(\ref{initial_flow}), 
%    (\ref{gst_left}),  (\ref{lbound}), (\ref{upbound}) 
    and objective function (\ref{all_coeff}).
Let BB (from Basic Binary) denote the same model except that constraint (\ref{reals}) is substituted by its binary variant, i.e. 

             \begin{equation}\label{binary_vertices}
      \forall v\in V: i_v  \in \{0,1\} \mbox{ and } s_v \in \{0,1\} \mbox{ and } t_v \in \{0,1\}.
             \end{equation}
             
%According to Theorem~\ref{vertices_are_binary}, the models BB and BR are equivalent in quality of the results.  We are interested here in their running time behavior.  
We study as well the linear programming relaxation of BR (denoted by BR$_{LP}$) where the  binary constraints (\ref{binary_edges}) are substituted by
   \begin{equation}\label{real_edges}
    \forall e \in O:  0 \leq x_e \leq 1  \mbox{ and } \forall e \in L:  0 \leq g_e \leq 1.
   \end{equation}
   
     Furthermore, let us omit from BR model  everything  that relates to 
 mate-pairs distances (i.e. the last term  in the objective function, as well the  constraints (\ref{gst_left}), (\ref{lbound}) and (\ref{upbound})). 
  We call this model LP (Longest Path)  since it simply targets finding the longest path in the contig graph. 
  Any solution of this model can be extended to a solution of BR model by a completion $g_e = 0 ~\forall e \in L$. Its optimal value yields a lower bound for the main model BR. 
  

  %  The results obtained by each one of these models are presented in  Table~\ref{tab:tools1}. From these results we first 
By experimentally analyzing these models   
 we observe that, as expected, the model BR significantly outperforms BB in running time, both models being equivalent in quality of the results.. 
\forLaterVersions{With respect to the quality of the obtained results, the results with the relaxed models are very encouraging.} 
The upper bounds computed by the linear relaxation  BR$_{LP}$ are extremely close to the
exact values computed by BR model. Furthermore, the quality of the lower
bound found by the longest path approach LP is also very good. Interestingly,
we observed for the given instances that this value is close %almost identical 
to the genome's size. We presume that the LP model can be used %successfully 
for predicting the length of the genome when it is unknown.

}
 %\input{revised_table3}    
 
\forLaterVersions{  

In order to tackle bigger instances and to make the previous models scalable
 we develop in the sequel the so called multistep step approach. To this end we first classify the contigs as big and small.   The contig  $v$ is called big if $w_v > l_e  ~ \forall e \in L$ holds. All other contigs  are called small.  The set of vertices $V$ is hence the union of big ($B$)  and small ($S$)  contigs, i.e. $V=B\cup S$. Big contigs are visualized as red vertices, while small--as blue  vertices. Moreover, we visualize overlaps as dashed  arrows, while links--as normal arrows  (see Fig. \ref{fig:Atropa_graph}).  The graph $G_R=(B,L)$ will be called reduced config graph--an instance of such a graph is represented on Fig. (\ref{fig:reduced_Atropa_graph}). Amongst the interesting features of the reduced
contig graph we would like to mention:
\begin{itemize}
\item The read coverage of big contigs can be considered more reliable compared to the one of small contigs which is frequently noisy.
\item The information provided by the big contigs combined with the mate-pair relationships can be very effective for resolving repeated sequences and for obtaining a more complete picture of the genome.
\item It is highly probable that the reduced graph contains the essential (the biggest portion) of the genome.
\item The reduced graph is notably smaller compared to the original contig graph and this is extremely beneficial for the performance of our approach.
\end{itemize}

The above has been confirmed experimentally through the results of the so called multistep step approach that we describe bellow. The first step of this approach consists in applying the LP model to the graph $G_R=(B,L)$ (i.e. we
search for the longest path in the reduced graph while setting $l_e=0 ~   \forall e \in L$). We obtain in this manner a lower bound and a partial solution which permits 
to set suitable values for all big contigs. The results obtained from this step are
reported as RG (Reduced Graph) in Table~\ref{tab:tools1}.
This partial scaffolding is further completed by gap filling as explained below.
We apply the BR model to the original graph G, but this time we "help" the
solver by setting some input parameters, namely: (i) a  partial solution (the values for all big
contigs) that has been  found  previously  in the reduced graph, (ii)  the value
computed by LP which is used as a lower bound. The results obtained in this
way are reported as BR$_P$ in Table~\ref{tab:tools1}. In this way the multistep approach consists
of 3 steps: LP, RG and BR$_P$ (see Table~\ref{tab:tools1}). This approach was very effective in the case of  
 Aethionema Cordifolium genome.
 }     

      


 \section{Conclusion}
 
 We developed and tested  algorithms for scaffolding  and gap filling phases based on a version of the longest path problem and MILP representation. Our algorithms significantly outperform three of the best known scaffolding algorithms with respect to the quality of the scaffolds. Regardless of that, we consider the current results as a work in progress. The biggest challenge is to extend the method to much bigger genomes. We plan to use some additional ideas and careful implementation to increase the scalability \forLaterVersions{of the methods} without sacrificing the accuracy of the results.            


 
\bibliographystyle{splncs03}
\bibliography{scaffoldingbiblio}

%\newpage\appendix
%\noindent{\large \bf Appendix}\\[2ex]
%
%\section{Proof of Theorem \ref{vertices_are_binary}}
%    \begin{proof} 
% Let us analyse the four possibles cases deduced from   (\ref{output-edges}).
% Denote $S^+= \sum_{e\in A^{+}(v)} x_e$ and $S^-=\sum_{e\in A^{-}(v)} x_e$.
%  \begin{enumerate}
%  \item[i)] $ S^+ = 0$ and $ S^- = 0$. 
%  
%  In this case  it follows  from (\ref{reals}) that  $s_v=i_v=t_v=0$. 
%    \item[ii)] $ S^+ = 1$ and $ S^- = 1$.
% 
%The above is equivalent to  $s_v + i_v = 1$ and  $t_v + i_v = 1$, which leads to $s_v=t_v$.  Moreover, from (\ref{at_most_one}) we have $s_v=t_v=0$ and $i_v=1$.
%    
%\item[iii)] $ S^+ = 1$ and $S^- = 0$.
%
%The above is equivalent to  $s_v + i_v = 1$ and  $t_v + i_v = 0$, which leads to $s_v-t_v=1$.  Hence, from (\ref{reals}), we have $t_v=0$ and therefore   $s_v=1$ and $i_v=0$.
%
%
%\item[iv)] $ S^+ = 0$ and $ S^- = 1$.
%
%This case is analogous to iii) and we have   $s_v=i_v=0$ and $t_v=1$. 
%  \end{enumerate} 
%    \end{proof}
%    
%    
%\newpage

 % \input{revised_table3}    
\end{document}